\documentclass[]{article}
\usepackage[inline]{asymptote}
\begin{document}

When we put (vertically) large expressions inside of parentheses (or brackets, or curly braces, etc.), the parentheses don't resize to fit the expression and instead remain relatively small. For instance, $$f(x) = \pi(\frac{\sqrt{x}}{x-1})$$ comes out as
\begin{figure}[!ht]
\centering	\begin{asy}
size(7cm,7cm,IgnoreAspect);

// Un premier tableau d'entiers de taille (3,4) rempli "manuellement"
int[][] x={{2,3,5,7},
{11,13,17,19},
{23,29,31,37}};
// Un second tableau d'entiers de taille (7,5) rempli avec une boucle
int[][] y=new int[7][5];
for (int i=0; i<7; ++i) 
for (int j=0; j<5; ++j) 
y[i][j]=i+j; 

// Affichage du tableau x
for (int i=0; i<3; ++i) 
for (int j=0; j<4; ++j) 
label(format("%i",x[i][j]),(j,-i));
// Affichage du tableau y
for (int i=0; i<7; ++i) 
for (int j=0; j<5; ++j) 
label(format("%i",y[i][j]),(j,-i-4));

shipout(bbox(.5cm,Fill(white)));
\end{asy}
\caption{wwwwww}
\end{figure}
When we put (vertically) large expressions inside of parentheses (or brackets, or curly braces, etc.), the parentheses don't resize to fit the expression and instead remain relatively small. For instance, $$f(x) = \pi(\frac{\sqrt{x}}{x-1})$$ comes out as


\end{document}

                                             